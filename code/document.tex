\documentclass[9pt]{amsart}
\usepackage[utf8]{inputenc}
\usepackage{amsmath, amssymb, amsthm}
\usepackage{mathrsfs, graphicx, tikz}
\usepackage[left=3cm, right=3cm, bottom=3.4cm]{geometry}
\usepackage{hyperref}
\usepackage{fancyhdr}

% Custom Operators
\DeclareMathOperator{\del}{Del}
\DeclareMathOperator{\ind}{Ind}

% Custom qed symbol
\renewcommand{\qedsymbol}{\emph{(end of proof)}}

% Theorem Styles
\theoremstyle{plain}
\newtheorem{theorem}{Theorem}
\renewcommand{\thetheorem}{\Roman{theorem}}

\newtheorem{proposition}{Proposition}
\renewcommand{\theproposition}{\Roman{proposition}}

\newtheorem{result}{Result}
\renewcommand{\theresult}{\Roman{result}}

\newtheorem{lemma}{Lemma}
\renewcommand{\thelemma}{\Roman{lemma}}

\newtheorem{conjecture}{Conjecture}
\renewcommand{\theconjecture}{\Roman{conjecture}}

% Title and Author
\title{\textbf{On the Oscillating Set Theorem for Convergence and Divergence}}
\author{Jayant Sharma}
\date{\today}

% Fancy Header
\pagestyle{fancy}
\fancyhf{}
\fancyhead[L]{\emph{On the Oscillating Set Theorem for Convergence and Divergence, Jayant Sharma}}
\fancyhead[R]{\thepage}
\renewcommand{\headrulewidth}{0.4pt}
\renewcommand{\footrulewidth}{0pt}

\begin{document}

\maketitle

\begin{abstract}
Let $U$ be an increasing infinite set. A subset $A \subseteq U$ is called an \emph{oscillating subset} of $U$ if the following condition holds:
\[
\max \left( \del\left( \ind(U, A) \right) \right) < \infty
\]
Here, the operator $\ind(U, A)$ returns the set of indices corresponding to the elements of $U$ that constitute the subset $A$. The operator $\del(k)$, defined on any finite or infinite ordered set $k$, returns the set of first-order differences between successive elements. That is, for a set
\[
k = \{ a_1, a_2, a_3, \dots, a_n \},
\]
we define
\[
\del(k) = \{ a_2 - a_1,\, a_3 - a_2,\, \dots,\, a_n - a_{n-1} \}.
\]
The operator $\max$ denotes the maximum of a set, i.e., the least upper bound of a set.

Furthermore, for any infinite set $K$, we define the function
\[
\tau(K) = \sum_{a_i \in K} \frac{1}{a_i},
\]
which captures the harmonic weight of the set $K$.

We aim to establish that if $A^\circ \subseteq A$ is an oscillating subset, then the convergence or divergence of $\tau(A)$ is equivalent to that of $\tau(A^\circ)$. In other words,
\[
\tau(A) < \infty \iff \tau(A^\circ) < \infty, \quad \text{and} \quad \tau(A) = \infty \iff \tau(A^\circ) = \infty.
\]
This result suggests that the oscillatory structure of a subset preserves the asymptotic harmonic behavior of the original set.
\end{abstract}

\tableofcontents

\section{Preliminaries}

We define a subset $A^{\circ} \subseteq A$ to be an \emph{oscillating subset} of $A$ if it satisfies the following condition:
\begin{align}
\max \left( \del\left( \ind(U, A) \right) \right) < \infty
\end{align}

Here, the operator $\ind(U, A)$ returns the set of indices in $U$ corresponding to the elements that form the subset $A$. The operator $\del(k)$, defined on any finite or infinite ordered set $k$, returns the set of first-order differences between successive elements. That is, for a set
\[
k = \{ a_1, a_2, a_3, \dots, a_n \},
\]
we define
\[
\del(k) = \{ a_2 - a_1,\, a_3 - a_2,\, \dots,\, a_n - a_{n-1} \}.
\]

The operator $\max$ denotes the maximum element of a set. When a maximum exists, it is the largest element in the set; otherwise, the supremum may be considered.

\vspace{1em}

For convenience, we define the function $\tau(A)$ for any infinite set $A \subseteq \mathbb{N}$ as the sum of the reciprocals of its elements:
\begin{align}
\tau(A) = \sum_{a_i \in A} \frac{1}{a_i}
\end{align}
If the subset $A^{\circ} \subseteq A$ is an oscillating subset, then it can be said about the set $A^{\circ}$ has a good density in the set $A$.
This motivates the question if the convergence and divergence of $\tau(A)$ is related to $\tau(A^{\circ})$?

Hence we state our theorem as,
\begin{theorem}[Oscillating Set Theorem]
If $A^{\circ}$ is an oscillating subset of $A$, then 
\[
\tau(A) < \infty \iff \tau(A^\circ) < \infty, \quad \text{and} \quad \tau(A) = \infty \iff \tau(A^\circ) = \infty
\]
\end{theorem}

This theorem highlights how the oscillating subset preserves the convergence or divergence of the superset.

\section{Investigating the Oscillating Set Theorem}

To investigate \emph{Theorem 1}, we begin by proposing the following proposition:

\begin{proposition}
Let $A$ be an increasing infinite set, and let $A' \subseteq A$ be a subset whose indices within $A$ form an arithmetic progression. Then,
\[
\tau(A) < \infty \iff \tau(A') < \infty, \quad \text{and} \quad \tau(A) = \infty \iff \tau(A') = \infty.
\]
\end{proposition}

\begin{proof}
We first consider the case where $\tau(A') = \infty$. Since $A' \subseteq A$, this directly implies $\tau(A) = \infty$.

For the converse, assume $\tau(A) = \infty$ but $\tau(A') < \infty$. Let the indices of elements of $A$ that form $A'$ follow an arithmetic progression with common difference $d \in \mathbb{N}$. Then, we express $A'$ as:
\[
A' = \lim_{n \to \infty} \{ a_k,\, a_{k+d},\, a_{k+2d},\, \dots,\, a_{k+(n-1)d} \}.
\]

Now define, for each $t \in \{1, 2, \dots, d-1\}$, the shifted subsets:
\[
A_t = \lim_{n \to \infty} \{ a_{k+t},\, a_{k+t+d},\, a_{k+t+2d},\, \dots,\, a_{k+t+(n-1)d} \}.
\]

Each $A_t \subseteq A$, and together with $A_0 := A'$, they partition the entire set $A$. That is,
\[
A = \bigcup_{t=0}^{d-1} A_t.
\]

Since $A$ is an increasing sequence of positive integers and $\tau(A_0) = \tau(A') < \infty$ by assumption, and because each $A_t$ has elements strictly larger than those in $A_0$, each corresponding harmonic sum $\tau(A_t)$ must also converge. Thus,
\[
\tau(A) = \sum_{t=0}^{d-1} \tau(A_t) < \infty,
\]
which contradicts our assumption that $\tau(A) = \infty$. Hence, our assumption must be false, and so $\tau(A') = \infty$.

Similarly, if $\tau(A') < \infty$, then all $\tau(A_t)$ are finite by monotonicity of $A$, and thus $\tau(A) = \sum_{t=0}^{d-1} \tau(A_t) < \infty$.

Finally, the case $\tau(A) < \infty \Rightarrow \tau(A') < \infty$ follows trivially since $A' \subseteq A$.

Thus, all four equivalences are verified, completing the proof.
\end{proof}

By \emph{Proposition 1}, if $A^\circ$ is an oscillating subset of $A$ with first element $a_p$, then at least one gap between consecutive indices must occur infinitely often due to condition~(1). Based on this, we can construct a subset $P \subseteq A$ defined as:
\begin{align}
P = \lim_{n \to \infty} \{ a_p,\, a_{p+d},\, a_{p+2d},\, \dots,\, a_{p+(n-1)d} \}
\end{align}
where $d \in \mathbb{N}$, and
\begin{align}
d > \max\left( \del\left( \ind(A, P) \right) \right)
\end{align}

Assume now that $\tau(A^\circ) < \infty$. Since both $P$ and $A^\circ$ start from the same element and, by $(3)$, the growth rate of $\tau(P)$ exceeds that of $\tau(A^\circ)$, it follows that $\tau(P) < \infty$. Furthermore, since $P \subseteq A$ and its indices form an arithmetic progression, \emph{Proposition 1} implies that $\tau(A) < \infty$.

Conversely, assume $\tau(A) < \infty$. Then, as $A^\circ \subseteq A$, it directly follows that $\tau(A^\circ) < \infty$.

Now consider the case where $\tau(A) = \infty$. Then, by \emph{Proposition 1}, we must have $\tau(P) = \infty$. Since $\tau(P)$ grows faster than $\tau(A^\circ)$, it follows that $\tau(A^\circ) = \infty$.

Finally, if $\tau(A^\circ) = \infty$, then since $A^\circ \subseteq A$, we conclude $\tau(A) = \infty$.

Hence, all four possible cases have been verified, completing the proof of \emph{Theorem 1}.

\section{Results}

Throughout this paper, we introduced and explored the concept of \textbf{Oscillating Subsets}. These subsets possess sufficient structural density within their supersets to preserve the convergence or divergence of the harmonic sum—that is, the sum of reciprocals of the elements of the superset.

We formally defined oscillating subsets as follows:

\begin{result}
A subset $A^\circ$ is called an oscillating subset of $A$ when $A^\circ \subseteq A$ and
\[
\max \left( \del\left( \ind(A, A^\circ) \right) \right) < \infty.
\]
Here, the operator $\ind(A, A^\circ)$ returns the set of indices in $A$ corresponding to the elements that form the subset $A^\circ$. The operator $\del(k)$, defined on any finite or infinite ordered set $k$, returns the set of first-order differences between successive elements. The operator $\max$ denotes the maximum of a set; if a maximum does not exist, the supremum is considered instead.
\end{result}

We then established the following central result:

\begin{result}[Oscillating Set Theorem]
Let $A$ be an increasing infinite set, and $A^\circ \subseteq A$ an oscillating subset. Then the convergence or divergence of the sum of reciprocals over one of the sets implies the same for the other. That is,
\[
\tau(A) < \infty \iff \tau(A^\circ) < \infty \quad \text{and} \quad \tau(A) = \infty \iff \tau(A^\circ) = \infty.
\]
\end{result}

This completes our investigation and affirms the fundamental harmonic preservation property of oscillating subsets.


\nocite{*}

\bibliographystyle{plain}
\bibliography{document}

\end{document}
