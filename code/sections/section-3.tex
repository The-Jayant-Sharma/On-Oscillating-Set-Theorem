\section{Results}

Throughout this paper, we introduced and explored the concept of \textbf{Oscillating Subsets}. These subsets possess sufficient structural density within their supersets to preserve the convergence or divergence of the harmonic sum—that is, the sum of reciprocals of the elements of the superset.

We formally defined oscillating subsets as follows:

\begin{result}
A subset $A^\circ$ is called an oscillating subset of $A$ when $A^\circ \subseteq A$ and
\[
\max \left( \del\left( \ind(A, A^\circ) \right) \right) < \infty.
\]
Here, the operator $\ind(A, A^\circ)$ returns the set of indices in $A$ corresponding to the elements that form the subset $A^\circ$. The operator $\del(k)$, defined on any finite or infinite ordered set $k$, returns the set of first-order differences between successive elements. The operator $\max$ denotes the maximum of a set; if a maximum does not exist, the supremum is considered instead.
\end{result}

We then established the following central result:

\begin{result}[Oscillating Set Theorem]
Let $A$ be an increasing infinite set, and $A^\circ \subseteq A$ an oscillating subset. Then the convergence or divergence of the sum of reciprocals over one of the sets implies the same for the other. That is,
\[
\tau(A) < \infty \iff \tau(A^\circ) < \infty \quad \text{and} \quad \tau(A) = \infty \iff \tau(A^\circ) = \infty.
\]
\end{result}

This completes our investigation and affirms the fundamental harmonic preservation property of oscillating subsets.
